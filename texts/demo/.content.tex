
% begin of MD document
\mktsShowpar\par
\end{multicols}
\section{MKTS Regions \cn{中國皇帝}
}
\begin{multicols}{2}To indicate the start of an MKTS-MD Region, place a triple at-sign {\mktsStyleCode{}@@@}
at the start of a line, immediately followed by a command name such as
{\mktsStyleCode{}keep-lines} or {\mktsStyleCode{}single-column}. The end of a region is indicated by a
triple at-sign without a command name. Nested regions are possible; for example,
you can put a {\mktsStyleCode{}@@@keep-lines} region inside a {\mktsStyleCode{}@@@single-column} region as
done here:\mktsShowpar\par
\end{multicols}
\section{MKTS Regions \cn{中國皇帝}
}
\begin{multicols}{2}To indicate the start of an MKTS-MD Region, place a triple at-sign {\mktsStyleCode{}@@@}
at the start of a line, immediately followed by a command name such as
{\mktsStyleCode{}keep-lines} or {\mktsStyleCode{}single-column}. The end of a region is indicated by a
triple at-sign without a command name. Nested regions are possible; for example,
you can put a {\mktsStyleCode{}@@@keep-lines} region inside a {\mktsStyleCode{}@@@single-column} region as
done here:\mktsShowpar\par
Here are some formulas:
\mktsShowpar\par
\begingroup\obeyalllines{}{\mktsStyleCode{}u-cjk/4e36}  \cn{丶}   \cnxJzr{}
{\mktsStyleCode{}u-cjk/4e37}  \cn{丷}   \cnxJzr{}\cn{丶}\cnxJzr{}
{\mktsStyleCode{}u-cjk/4e38}  \cn{丸}   \cnxJzr{}\cn{九丶}
{\mktsStyleCode{}u-cjk/4e39}  \cn{丹}   \cnxJzr{}\cnxBabel{⺆}\cnxJzr{}\cn{丶一}
{\mktsStyleCode{}u-cjk/4e3a}  \cn{为}   \cnxJzr{}\cn{丶}\cnxJzr{}\cn{力丶}
{\mktsStyleCode{}u-cjk/4e3b}  \cn{主}   \cnxJzr{}\cn{丶王}
{\mktsStyleCode{}u-cjk/4e3b}  \cn{主}   \cnxJzr{}\cn{亠土}
{\mktsStyleCode{}u-cjk/4e3c}  \cn{丼}   \cnxJzr{}\cn{井丶}

{\mktsStyleCode{}u-cjk-xb/250b7}  \cnxb{𥂷}   \cnxJzr{}\cnxJzr{}\cn{告巨皿}
{\mktsStyleCode{}u-cjk-xb/250b8}  \cnxb{𥂸}   \cnxJzr{}\cn{楊皿}
\endgroup{}At this point, a line consisting of a triple at-sign {\mktsStyleCode{}@@@}
indicates the end of the {\mktsStyleCode{}keep-lines} region; since the
{\mktsStyleCode{}single-column} region is still active, however, {\mktsStyleItalic{}this
paragraph runs across the entire width} of the documents text
area.
Now a {\mktsStyleCode{}\}single-column} MKTS/MD event has been encountered
that was triggered by a triple-at command in the manuscript;
accordingly, typesetting is reverted back to multi-column mode,
which is why you can see this paragraph set in two columns.\mktsShowpar\par
\mktsShowpar\par

% end of MD document
