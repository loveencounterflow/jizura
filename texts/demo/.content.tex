
% begin of MD document

\chapter{This is a Demonstration \cn{中國皇帝}
}

\section{A Section Title 1 \cn{中國皇帝}
}


14{\mktsFontfileOptima‰}\par

yadda yadda yadda yadda yadda yadda yadda yadda yadda yadda
yadda yadda yadda yadda yadda yadda yadda yadda yadda yadda
yadda yadda yadda yadda yadda yadda yadda yadda yadda yadda
yadda yadda yadda yadda yadda yadda yadda yadda yadda yadda\par
\section{A Section Title 2 \cn{中國皇帝}
}


yadda yadda yadda yadda yadda yadda yadda yadda yadda yadda
yadda yadda yadda yadda yadda yadda yadda yadda yadda yadda
yadda yadda yadda yadda yadda yadda yadda yadda yadda yadda
yadda yadda yadda yadda yadda yadda yadda yadda yadda yadda\par
\section{A Section Title 3 \cn{中國皇帝}
}


yadda yadda yadda yadda yadda yadda yadda yadda yadda yadda
yadda yadda yadda yadda yadda yadda yadda yadda yadda yadda
yadda yadda yadda yadda yadda yadda yadda yadda yadda yadda
yadda yadda yadda yadda yadda yadda yadda yadda yadda yadda\par
\section{MKTS Regions \cn{中國皇帝}
}


To indicate the start of an MKTS-MD Region, place a triple at-sign {\mktsFontfileSourcecodeproregular{}@@@}
at the start of a line, immediately followed by a command name such as
{\mktsFontfileSourcecodeproregular{}keep-lines} or {\mktsFontfileSourcecodeproregular{}single-column}. The end of a region is indicated by a
triple at-sign without a command name. Nested regions are possible; for example,
\null\newpage{}you can put a {\mktsFontfileSourcecodeproregular{}@@@keep-lines} region inside a {\mktsFontfileSourcecodeproregular{}@@@single-column} region as
done here:\par

Here are some formulas:
\begingroup\obeyalllines{}{\mktsFontfileSourcecodeproregular{}u-cjk/4e36}  \cn{丶}   \cnxJzr{}
{\mktsFontfileSourcecodeproregular{}u-cjk/4e37}  \cn{丷}   \cnxJzr{}\cn{丶}\cnxJzr{}
{\mktsFontfileSourcecodeproregular{}u-cjk/4e38}  \cn{丸}   \cnxJzr{}\cn{九丶}
{\mktsFontfileSourcecodeproregular{}u-cjk/4e39}  \cn{丹}   \cnxJzr{}\cnxBabel{⺆}\cnxJzr{}\cn{丶一}
{\mktsFontfileSourcecodeproregular{}u-cjk/4e3a}  \cn{为}   \cnxJzr{}\cn{丶}\cnxJzr{}\cn{力丶}
{\mktsFontfileSourcecodeproregular{}u-cjk/4e3b}  \cn{主}   \cnxJzr{}\cn{丶王}
{\mktsFontfileSourcecodeproregular{}u-cjk/4e3b}  \cn{主}   \cnxJzr{}\cn{亠土}
{\mktsFontfileSourcecodeproregular{}u-cjk/4e3c}  \cn{丼}   \cnxJzr{}\cn{井丶}\par

{\mktsFontfileSourcecodeproregular{}u-cjk-xb/250b7}  \cnxb{𥂷}   \cnxJzr{}\cnxJzr{}\cn{告巨皿}
{\mktsFontfileSourcecodeproregular{}u-cjk-xb/250b8}  \cnxb{𥂸}   \cnxJzr{}\cn{楊皿}\par

\endgroup{}\par

At this point, a line consisting of a triple at-sign {\mktsFontfileSourcecodeproregular{}@@@}
indicates the end of the {\mktsFontfileSourcecodeproregular{}keep-lines} region; since the
{\mktsFontfileSourcecodeproregular{}single-column} region is still active, however, \textit{this
paragraph runs across the entire width} of the documents text
area.\par

xxxx\par

\par
\chapter{Another Demonstration
}


yadda yadda yadda yadda yadda yadda yadda yadda yadda yadda
yadda yadda yadda yadda yadda yadda yadda yadda yadda yadda
yadda yadda yadda yadda yadda yadda yadda yadda yadda yadda
yadda yadda yadda yadda yadda yadda yadda yadda yadda yadda
yadda yadda yadda yadda yadda yadda yadda yadda yadda yadda
yadda yadda yadda yadda yadda yadda yadda yadda yadda yadda
yadda yadda yadda yadda yadda yadda yadda yadda yadda yadda\par
--------------


yadda yadda yadda yadda yadda yadda yadda yadda yadda yadda
yadda yadda yadda yadda yadda yadda yadda yadda yadda yadda
yadda yadda yadda yadda yadda yadda yadda yadda yadda yadda
yadda yadda yadda yadda yadda yadda yadda yadda yadda yadda
yadda yadda yadda yadda yadda yadda yadda yadda yadda yadda
yadda yadda yadda yadda yadda yadda yadda yadda yadda yadda
yadda yadda yadda yadda yadda yadda yadda yadda yadda yadda\par
**************


yadda yadda yadda yadda yadda yadda yadda yadda yadda yadda
yadda yadda yadda yadda yadda yadda yadda yadda yadda yadda
yadda yadda yadda yadda yadda yadda yadda yadda yadda yadda
yadda yadda yadda yadda yadda yadda yadda yadda yadda yadda
yadda yadda yadda yadda yadda yadda yadda yadda yadda yadda
yadda yadda yadda yadda yadda yadda yadda yadda yadda yadda
yadda yadda yadda yadda yadda yadda yadda yadda yadda yadda\par
\chapter{Regions
}

\section{Keep-Lines Regions
}

--------------


A-before
\begingroup\obeyalllines{}A-within
A-within
A-within
\endgroup{}A-after\par

\%∆∆∆end\par
--------------


B-before\par

\begingroup\obeyalllines{}B-within
B-within
B-within
\endgroup{}\par

B-after\par
--------------


C-before
\begingroup\obeyalllines{}\par

C-within
C-within
C-within\par

\endgroup{}C-after\par
--------------


D-before\par

\begingroup\obeyalllines{}\par

D-within
D-within
D-within\par

\endgroup{}\par

D-after\par
% end of MD document
