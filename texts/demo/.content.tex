
% begin of MD document
\mktsShowpar\par
a\TeX{}b{\mktsStyleBold{}MKTS}c**DEF**d\mktsShowpar\par
{\mktsStyleCode{}a\textbackslash{}<<!TEX>>b<<!MKTS>>c**DEF**d}\mktsShowpar\par
Here is a footnote reference, and another,
and a third one.\mktsShowpar\par
Here is an inline note.\mktsShowpar\par
{\mktsStyleBold\color{violet}{%
\mktsStyleSymbol█}(footnote {\mktsStyleSymbol}}Here is the footnote.\mktsShowpar\par
{\mktsStyleBold\color{violet}{%
\mktsStyleSymbol}footnote) {\mktsStyleSymbol█}}{\mktsStyleBold\color{violet}{%
\mktsStyleSymbol█}(footnote {\mktsStyleSymbol}}Here’s one with multiple blocks.\mktsShowpar\par
Subsequent paragraphs are indented to show that they
belong to the previous footnote.\mktsShowpar\par
{\mktsStyleBold\color{violet}{%
\mktsStyleSymbol}footnote) {\mktsStyleSymbol█}}{\mktsStyleBold\color{violet}{%
\mktsStyleSymbol█}(footnote {\mktsStyleSymbol}}Third footnote.\mktsShowpar\par
{\mktsStyleBold\color{violet}{%
\mktsStyleSymbol}footnote) {\mktsStyleSymbol█}}{\mktsStyleBold\color{violet}{%
\mktsStyleSymbol█}(footnote {\mktsStyleSymbol}}Inlines notes are easier to write, since
you don't have to pick an identifier and move down to type the
note.\mktsShowpar\par
{\mktsStyleBold\color{violet}{%
\mktsStyleSymbol}footnote) {\mktsStyleSymbol█}}
% end of MD document
