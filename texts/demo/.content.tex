\mktsComment{./.}
% begin of MD document
\mktsComment{\{multi-column\}}\begin{multicols}{2}\mktsShowpar\par
\mktsComment{[h2] \{multi-column\}}\end{multicols}
\section{MKTS Regions \cn{中國皇帝}\mktsComment{\{multi-column\}}
}
\begin{multicols}{2}To indicate the start of an MKTS-MD Region, place a triple at-sign \mktsComment{(code) \{multi-column\}}{\mktsStyleCode{}@@@\mktsComment{\{multi-column\}}}
at the start of a line, immediately followed by a command name such as
\mktsComment{(code) \{multi-column\}}{\mktsStyleCode{}keep-lines\mktsComment{\{multi-column\}}} or \mktsComment{(code) \{multi-column\}}{\mktsStyleCode{}single-column\mktsComment{\{multi-column\}}}. The end of a region is indicated by a
triple at-sign without a command name. Nested regions are possible; for example,
you can put a \mktsComment{(code) \{multi-column\}}{\mktsStyleCode{}@@@keep-lines\mktsComment{\{multi-column\}}} region inside a \mktsComment{(code) \{multi-column\}}{\mktsStyleCode{}@@@single-column\mktsComment{\{multi-column\}}} region as
done here:\mktsShowpar\par
\mktsComment{[h2] \{multi-column\}}\end{multicols}
\section{MKTS Regions \cn{中國皇帝}\mktsComment{\{multi-column\}}
}
\begin{multicols}{2}To indicate the start of an MKTS-MD Region, place a triple at-sign \mktsComment{(code) \{multi-column\}}{\mktsStyleCode{}@@@\mktsComment{\{multi-column\}}}
at the start of a line, immediately followed by a command name such as
\mktsComment{(code) \{multi-column\}}{\mktsStyleCode{}keep-lines\mktsComment{\{multi-column\}}} or \mktsComment{(code) \{multi-column\}}{\mktsStyleCode{}single-column\mktsComment{\{multi-column\}}}. The end of a region is indicated by a
triple at-sign without a command name. Nested regions are possible; for example,
you can put a \mktsComment{(code) \{multi-column\}}{\mktsStyleCode{}@@@keep-lines\mktsComment{\{multi-column\}}} region inside a \mktsComment{(code) \{multi-column\}}{\mktsStyleCode{}@@@single-column\mktsComment{\{multi-column\}}} region as
done here:\mktsShowpar\par
\mktsComment{\{multi-column\} \{single-column\}}\end{multicols}Here are some formulas:
\mktsShowpar\par
\mktsComment{\{keep-lines\} \{multi-column\} \{single-column\}}\begingroup\obeyalllines{}\mktsComment{(code) \{keep-lines\} \{multi-column\} \{single-column\}}{\mktsStyleCode{}u-cjk/4e36\mktsComment{\{keep-lines\} \{multi-column\} \{single-column\}}}  \cn{丶}   \cnxJzr{}
\mktsComment{(code) \{keep-lines\} \{multi-column\} \{single-column\}}{\mktsStyleCode{}u-cjk/4e37\mktsComment{\{keep-lines\} \{multi-column\} \{single-column\}}}  \cn{丷}   \cnxJzr{}\cn{丶}\cnxJzr{}
\mktsComment{(code) \{keep-lines\} \{multi-column\} \{single-column\}}{\mktsStyleCode{}u-cjk/4e38\mktsComment{\{keep-lines\} \{multi-column\} \{single-column\}}}  \cn{丸}   \cnxJzr{}\cn{九丶}
\mktsComment{(code) \{keep-lines\} \{multi-column\} \{single-column\}}{\mktsStyleCode{}u-cjk/4e39\mktsComment{\{keep-lines\} \{multi-column\} \{single-column\}}}  \cn{丹}   \cnxJzr{}\cnxBabel{⺆}\cnxJzr{}\cn{丶一}
\mktsComment{(code) \{keep-lines\} \{multi-column\} \{single-column\}}{\mktsStyleCode{}u-cjk/4e3a\mktsComment{\{keep-lines\} \{multi-column\} \{single-column\}}}  \cn{为}   \cnxJzr{}\cn{丶}\cnxJzr{}\cn{力丶}
\mktsComment{(code) \{keep-lines\} \{multi-column\} \{single-column\}}{\mktsStyleCode{}u-cjk/4e3b\mktsComment{\{keep-lines\} \{multi-column\} \{single-column\}}}  \cn{主}   \cnxJzr{}\cn{丶王}
\mktsComment{(code) \{keep-lines\} \{multi-column\} \{single-column\}}{\mktsStyleCode{}u-cjk/4e3b\mktsComment{\{keep-lines\} \{multi-column\} \{single-column\}}}  \cn{主}   \cnxJzr{}\cn{亠土}
\mktsComment{(code) \{keep-lines\} \{multi-column\} \{single-column\}}{\mktsStyleCode{}u-cjk/4e3c\mktsComment{\{keep-lines\} \{multi-column\} \{single-column\}}}  \cn{丼}   \cnxJzr{}\cn{井丶}

\mktsComment{(code) \{keep-lines\} \{multi-column\} \{single-column\}}{\mktsStyleCode{}u-cjk-xb/250b7\mktsComment{\{keep-lines\} \{multi-column\} \{single-column\}}}  \cnxb{𥂷}   \cnxJzr{}\cnxJzr{}\cn{告巨皿}
\mktsComment{(code) \{keep-lines\} \{multi-column\} \{single-column\}}{\mktsStyleCode{}u-cjk-xb/250b8\mktsComment{\{keep-lines\} \{multi-column\} \{single-column\}}}  \cnxb{𥂸}   \cnxJzr{}\cn{楊皿}
\mktsComment{\{multi-column\} \{single-column\}}\endgroup{}At this point, a line consisting of a triple at-sign \mktsComment{(code) \{multi-column\} \{single-column\}}{\mktsStyleCode{}@@@\mktsComment{\{multi-column\} \{single-column\}}}
indicates the end of the \mktsComment{(code) \{multi-column\} \{single-column\}}{\mktsStyleCode{}keep-lines\mktsComment{\{multi-column\} \{single-column\}}} region; since the
\mktsComment{(code) \{multi-column\} \{single-column\}}{\mktsStyleCode{}single-column\mktsComment{\{multi-column\} \{single-column\}}} region is still active, however, \mktsComment{(em) \{multi-column\} \{single-column\}}{\mktsStyleItalic{}this
paragraph runs across the entire width\mktsComment{\{multi-column\} \{single-column\}}} of the documents text
area.
\mktsComment{\{multi-column\}}\begin{multicols}{2}Now a \mktsComment{(code) \{multi-column\}}{\mktsStyleCode{}\}single-column\mktsComment{\{multi-column\}}} MKTS/MD event has been encountered
that was triggered by a triple-at command in the manuscript;
accordingly, typesetting is reverted back to multi-column mode,
which is why you can see this paragraph set in two columns.\mktsShowpar\par
\mktsComment{./.}\end{multicols}\mktsShowpar\par
\mktsComment{./.}
% end of MD document
