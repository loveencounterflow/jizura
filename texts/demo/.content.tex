
^^^^4e00^^^^007b^^^^007d

% begin of MD document
\begin{multicols}{2}
You can use {\mktsStyleCode{}<<\textbackslash{}{latex>>...<<latex\textbackslash{}}>>} or {\mktsStyleCode{}<<(latex>>...<<latex)>>} to directly insert LaTeX
code into your script; for example, you could
use {\mktsStyleCode{}<<(latex>>\textbackslash{}LaTeX<<latex)>>}
to obtain the \LaTeX{} logogram.
Observe we wrote {\mktsStyleCode{}<<(latex>>\textbackslash{}LaTeX\textbackslash{}{\textbackslash{}}<<latex)>>} here
(instead of {\mktsStyleCode{}<<(latex>>\textbackslash{}LaTeX<<latex)>>}){}
in order to preserve the space between the logogram itself and
the word ‘logogram’—MKTS will not intervene to make that happen
automatically, as a careful, scientific study has demonstrated
that this problem—preserving spaces following commands in a
general way that does not rely on parsing {\mktsStyleCode{}<<(latex>>\textbackslash{}LaTeX\textbackslash{}{\textbackslash{}}<<latex)>>}
source and is not going to muck with very deep
{\mktsStyleCode{}<<(latex>>\textbackslash{}TeX\textbackslash{}{\textbackslash{}}<<latex)>>}
internals—is NP-complete.\mktsShowpar\par
Another potential use is to {\color{red}COLORIZE!} your text, here done by inserting\mktsShowpar\par
\begingroup\obeyalllines\mktsStyleCode{}<<(latex>>\textbackslash{}{\textbackslash{}color\textbackslash{}{red\textbackslash{}}<<latex)>>
COLORIZE!
<<(latex>>\textbackslash{}}<<latex)>>
\endgroup{}(with or without the line breaks) into the script.
\end{multicols}xxx\mktsShowpar\par

% end of MD document
