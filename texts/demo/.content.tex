
% begin of MD document
xxx {\mktsStyleCode{}\\foo} bar\mktsShowpar\par
xxx {\mktsStyleCode{}\\foo}
bar\mktsShowpar\par
\begin{multicols}{2}
You can use {\mktsStyleCode{}<<\{latex>>...<<latex\}>>} or {\mktsStyleCode{}<<(latex>>...<<latex)>>} to directly insert LaTeX
code into your script; for example, you could
use {\mktsStyleCode{}<<(latex>>\\LaTeX<<latex)>>}
to obtain the \LaTeX{} logogram.
Observe that we had to write {\mktsStyleCode{}\\LaTeX\{\}} here instead of {\mktsStyleCode{}\\LaTeX} to preserve the space between the logogram itself and
the word ‘logogram’—MKTS will not intervene to make that happen
automatically, as a careful, scientific study has demonstrated
that this problem—preserving spaces following commands in a
general way that does not rely on parsing \LaTeX{}
source and is not going to muck with very deep
\TeX{}
internals—is NP-complete.\mktsShowpar\par
Another potential use of  is to {\color{red}COLORIZE!} your text, here done by inserting\mktsShowpar\par
\begingroup\obeyalllines\mktsStyleCode{}<<(latex>>\{\\color\{red\}<<latex)>>
COLORIZE!
<<(latex>>\}<<latex)>>
\endgroup{}(with or without the line breaks) into the script.
\end{multicols}xxx\mktsShowpar\par

% end of MD document
